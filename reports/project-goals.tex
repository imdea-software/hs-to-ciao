\documentclass[11pt]{article}

% Margins the way I like them
\usepackage[top=0.8in, bottom=0.8in, left=1.2in, right=1.2in]{geometry}

\usepackage[utf8]{inputenc}
\usepackage{xcolor, graphicx, tikz}

\usepackage[backend=biber, style=alphabetic]{biblatex}
\addbibresource{references.bib}

\setlength\parindent{0pt}

\usepackage[hidelinks]{hyperref}
\hypersetup{%
    colorlinks=true,
    linkcolor=black,
    citecolor=magenta
  }
  
\title{
  Haskell to Ciao: Translation of Haskell code for~Static~Resource~Analysis~in~CiaoPP}
\author{David Munuera Mazarro \\ \texttt{david.munuera@imdea.org}}
\date{}
\begin{document}
\pagenumbering{gobble}

\pagebreak

\begin{minipage}{1.0\linewidth}

  {\hspace{4.6cm}
    \includegraphics[scale=0.2]{hs-to-ciao.png}}

  \vspace{-1.3cm}

\maketitle
  
\end{minipage}

The goals of this project are:
\begin{itemize}
\item Creating a transpiler (Source-to-Source compiler) that takes Haskell code (or, more specifically, GHC Core code)
  and yields code for an equivalent Ciao Prolog program
\item Perform a static resource analysis
  running the CiaoPP engine over the Ciao code provided by the translation
\item Draw conclusions from the analysis results
  and take them to the original Haskell domain from which we started
\end{itemize}
The utmost objective of the project is to be able to take some Haskell code,
perhaps meeting some conditions imposed by the limitations of the translation,
and obtain resource usage properties inferred by
the CiaoPP engine accurately matching said Haskell code. \\

There are a few challenges posed by this task.
First of all, the translation has to be \textbf{correct},
because otherwise the results and conclusions taken from the analysis
would only be valid for the Ciao code resulting from the translation,
and not necessarily all the way back to the original code. In addition,
we should aim for the translation to be \textbf{as complete as possible} in order
to be able to analyze real, useful Haskell programs and not just
little example snippets.
But there may not be an immediate translation for certain basic Haskell features
into basic Ciao features (for example, Haskell's basic types are
different from Ciao's), and for every such situation we encounter,
we will either have to:

\begin{enumerate}
\item[a)] Find a workaround translation that remains compatible with CiaoPP, or
\item[b)] If the above proves to be difficult or not feasible,
  carry the burden into the transpiler and regard it as a limitation
\end{enumerate}

Moving back to the correctness of the translation. Ideally, the translation
should be formally proven to be correct. However, I do not currently have
the theoretical knowledge to do so, and the way I intend to approach
the project is by trial-and-error; using the transpiler to translate example programs,
check that they behave as they should, and if that's not the case, tweak the
transpiler as needed until the behavior of the translation matches that
of the original program.

% - Traducción de programas Haskell a claúsulas de Horn
% - Análisis de coste mediante CiaoPP, mejoras
% - Reflexión de los resultados en el programa original
% - Experimentos, generación de resultados
% - Documentación
\printbibliography

\end{document}

%%% Local Variables:
%%% mode: latex
%%% TeX-master: t
%%% TeX-command-extra-options: "-shell-escape"
%%% End: